\chapter{Literature Review} \label{litrev}
This chapter is about the literature review for this project. In this chapter, the relevant literature and theories will be reviewed. The literature review will examine the dangers of an inactive lifestyle and the benefits of exercising and the existing tools and technologies that have been proposed to solve the issue Reviewing the current research in human-computer interaction that involves exercise and health. What research has been done with health and exercise tracking with colleagues in work places and friends. The chapter looks at the social incentives, competition, cooperation and social influence that comes from fitness application. It looks at gamification, what it is, and what effect leaderboards have on users.

\section{HCI and promoting physical activity}
Ahtinen et al\cite{Ahtinen2009} did a study on wellness and the effect of using ubiquitous technologies and mobile phones to see the social interaction and information sharing in groups that had similar fitness goals. They found that using communication technology to spread wellness-related information and applications as gifts within a network of trusted of people. This gives the initial motivation and push to start using wellness applications and working towards better wellbeing. They also suggested that having peer-support from people with similar goals can increase the motivation to reach a better wellness level. One of the interviewed participants in the study noted that maybe there could be some option of also uploading some new[exercise] combinations which you have discovered, and you want to share with other users.

Sharing a fitness goal and activities with friends can increase the motivation for users and impact the general health of those that share them publicly. Announcing  commitments can catalyze support and accountability from existing social networks for health behavior change \cite{Munson2015}. However, they found that announcing your public fitness goal creates a selection effect that decreases the probability of making commitments (ibid).  By having private groups consisting of friends and families instead of sharing your fitness goals on social media could increase the probability of participating and increase the follow-through.

A study by Chen et al\cite{Chen2017} looked at the effect of using a leaderboard as a social incentive to track fitness data and researched the difference in interaction between workplace colleagues and chronically ill. The research showed that both groups were motivated by having a leaderboard for their fitness tracking. The healthy groups of participants saw the leaderboard as a competition whereas the chronically ill groups used it as a way of obtaining information from other patients. 

\section{Health tracking in work places}
Many companies have health incentives for their employees where they can get gym membership discounts and even monetary incentives as motivation for reaching a goal. A study by Chung et al\cite{Chung2017} researched the health tracking in work places that had wellness programs for their employees. The study suggests that sharing fitness data between the employees supported the fitness tracking. Having a collective goal such as walking 7000 steps every day was also beneficial to the fitness tracking.  individuals participating in team-based step-counting initiatives become accountable to each other for both the number of the steps they take and their reasons for walking. \cite{Buis2009}

A study done by Xipei et al\cite{Ren2018} explored this further to see if cooperative fitness tracking in work places encouraged physical activity. The study had participants that were divided in to groups of two and each group had a collective goal for daily steps. The study found that the participants were more active compared to the initial baseline week. The participants that were in the same office or proximity of each other did more physical activity than the participants that were in different offices. In addition, the participants that were closer would communicate more with each other and share information and knowledge of workouts. 

\section{Social incentives for fitness applications}
A paper by Chen and Pu\cite{Chen2017} looked at social incentives for mobile fitness applications where they developed a mobile game to understand how users interact in groups for competition or cooperation. The participants increased their physical activity significantly when paired up in the groups, however they found that “competition motivates dyads if they have equivalent performances and availabilities, but is likely to demotivate them if otherwise”\cite{Chen2017}. A better social incentive for a fitness application if the groups have different levels of performance is for them to cooperate with each other instead of competing.  “Cooperation setting is more likely to stimulate users to interact with each other via messages than competition settings”(ibid), having the users cooperate increases the interaction and possibly the knowledge sharing in between them.

Another social incentive is social influence, which consists of social comparison and social support.” In fitness apps, using leaderboards, people could easily check their relative performance positions”\cite{Zhou2018} for comparing themselves in social networks, and receiving social support by receiving positive feedback and comments.

Using cooperation as a social incentive, it is better to use strong ties consisting of friends, families or colleagues than strangers with weak ties. “Participants found community competition and the support from strong ties motivating, however, pairing up with weak ties was reported demotivating”\cite{Chen2016} Also suggesting that by creating a community effect for the competition will increase the cooperation and maybe have mentors that share knowledge.

\section{Gamification and fitness applications}
Gamification in fitness applications refers to the “the use of game design elements in nongame contexts” \cite{Deterding}. It is a popular strategy as a way of motivating users to adopt and use their application. A way of gamifying fitness applications may refer to the use of augmented reality where the users can run routes and pick up objects in the game, or just having an overall leaderboard for the people that have worked out the most in a period.

 “Gamification may be an effective means of targeting motivational components, and games may be effective at triggering individuals and increasing popularity of apps”\cite{Lister} and “external incentives are enough to sustain (health) behavior responses without using other components of games like problem solving, storytelling, and fantasy”(ibid).  For long term health change and motivation, the external incentives are enough, the study shows there is no need for augmented reality, 3d technology or storytelling.  The same study shows that gamification mostly works for easy physical activity like walking, “complex behaviors such as diet and physical activity”(ibid), which may lead to more knowledge and a better long term health change for the users.

\section{Comparison, competition and cooperation in fitness applications}
 Social support and social pressure positively influence user motivation\cite{Buis2009}, most fitness applications have features where the users are able to either compare, compete or cooperate with other users. Comparison in fitness applications is defined as “the design that facilitates benchmarking individual’s fitness performance with that of others, and hence provide an opportunity for enhanced motivation in target behaviors” \cite{Yoganathan2013}. The users can compare their own results with others, for example compare how much time spent doing a given task. Another popular feature is competition, which in fitness applications is “the design that motivate enhanced physical activity performance by leveraging human’s natural drive to compete” (ibid). This is usually done with a leaderboard where users are ranked on how much they have worked out or their time or how many steps they have done within a timeframe.  Lastly there is cooperation in fitness applications is “the design that motivates users to adopt physical activity behavior by leveraging an individual’s natural drive to cooperate” (ibid). Users can cooperate and work for a common goal, share information and knowledge and form social bonds with each other.

 

\section{Fitness tracking and devices}
People track their fitness activity in diary reports and blogs.\cite{Goodman} Tracking your activity is beneficial for fitness behaviour change \cite{mccormick2015psychological}, logging workouts to show the progress in terms of weights lifted as well as tracking other goals like losing or gaining weight helps the overall behaviour change. Many mobile fitness applications design revolves around tracking activity and goal setting where the aim is to increase the users physical activity \cite{consolvo2007conducting}. 
The majority of HCI research that investigates fitness
technology use and tracking is for the purpose of evaluating specific fitness technologies\cite{Patel2015}. 

People that use devices and fitness trackers often abandon devices because they do not fit with their conceptions of themselves, the data collected by the devices are perceived to not be useful, and device maintenance became unmanageable. \cite{Lazar2015}. Unless the goal of a person is to track their steps and activity, devices purchased often do not appear to map to goals\cite{Lazar2015}. Goals like becoming a better table tennis player, or obtaining knowledge from others. People perceive the data collected as not useful because they are not interested in the level of information the data gives them. Many people mention that the number of steps they take is not interesting\cite{Lazar2015}. 

The ability for users to set their own primary and secondary goals is key for technology-based health interventions to be successful\cite{inproceedings}

\section{Related work and fitness tracking applications}
There are several applications within the area of fitness and exercise. The applications allow users to obtain information and workouts, track their food consumption and daily exercise and even get personalized templates from personal trainers or buy training programs from fitness influencers. Some of the similar applications are:
Fitbit: “Without a tracker, the Fitbit app can count your steps (provided your carry your phone all day long), help you track the calories you consume, log your weight, and record other health information, such as blood pressure and glucose levels. If you do own a Fitbit tracker, the app is even easier to use because it logs a good amount of information about your activity automatically.”\cite{Duffy}
FitStar: “FitStar creates custom workouts for you based on your fitness level. You start by doing a few workouts with the app and you give it feedback as you go about which exercises were too tough, too easy, or just right. The app uses that information to create a routine that challenges you in all the right ways.”(ibid)
Lose it: “The free website and app Lose It!, designed for counting calories and logging exercise, can help you lose weight, especially if you tend to eat name-brand American foods. Lose It!, which has been around for years, has an incredibly strong community of supportive people to help you stick to your goals.”(ibid)
MyFitnessPal:” MyFitnessPal is a mobile app and website that gives you a wealth of tools for tracking what and how much you eat, and how many calories you burn through activity”(ibid)
Google Fit:” it's extremely serviceable and it's one of the better free fitness apps. It can do a lot of stuff. You can track your fitness using a point system as well as active minutes. The app also features fitness goal tracking, customized tips, and integration with a variety of other apps like Runkeeper, Strava, MyFitnessPal, and others. \cite{Hindy}
Gravitus: “Gravitus is the app for weight lifters. We live in a digital world and yet lifting is done completely offline. We're building the future of lifting. Enter the gym knowing exactly what you need to do. Record your workouts with a tool designed for speed. Celebrate your progress and break through plateaus. And connect with friends and others to stay motivated. Gravitus helps you reach your goals at the gym and have more fun doing it.”\cite{Gravitus}

