\chapter{Discussion}
In this chapter there is a discussion about the methodologies and the methods that were used, the development, design and limitations. The research questions are also discussed and answered at the end. 
\section{Methodologies}

\section{Design Science Research}
The design science research method was used throughout this research project. The eight design science research questions were used to be certain that the correct steps were followed during the development of the application. 
\begin{itemize}
    \item 1. What is the research question(design requirements)? 
\end{itemize}
 The research questions, goals   from section \ref{RQs} and design requirements from section \ref{requirements} were formulated to be relevant for the intended target group. The questions and goals were established in the early phases of the research project. This made it easy to go through the different design iterations with a clear focus on what questions needed to be solved. 
\begin{itemize}
    \item 2. What is the artifact? How is the artifact represented?
\end{itemize}
The artifact is a high-fidelity prototype called fit with friends. The artifact was designed and built with design principles in mind and the user needs and ideas established from the research project.  Previous chapter \ref{fitwithf} shows the features of the application.
\begin{itemize}
    \item 3. What design processes were used to build the artifact? 
\end{itemize}
Throughout the design cycle for this project there were different design processes used. Conceptual design (section \ref{concept} ), Interaction design principles (section \ref{interactiondesign} ) were used to build the prototypes and Nielsen`s heuristics from section \ref{nielsen} were utilized too.

\begin{itemize}
    \item 4. How are the artifact and the design processes grounded by the knowledge base? What,if any theories support the artifact design and the design process?
\end{itemize}
The artifact and design processes are grounded by the literature review that was conducted in chapter \ref{litrev}. Users and experts were interviewed to gather information and for the whole research project they  were used to evaluate the different iterations of the prototype. 
\begin{itemize}
    \item 5. What evaluations are performed during the internal design cycles? What design improvements are identified during each cycle?
\end{itemize}
The internal design cycles focused on rapid iterations and feedback to reach a satisfactory design suggestion. Different methods were used to evaluate the prototype, such as the usability testing (section \ref{usat} ) , SUS with experts (section \ref{suse} ) , SUS with users (section \ref{susu} ) and Nielsen`s heuristics (section \ref{heur} ) .
\begin{itemize}
    \item 6. How is the artifact introduced into the application environment and how is the field tested? What metrics are used to demonstrate artifact utility and improvement over previous artifact?
\end{itemize}
The artifact was field tested with a usability test in section \ref{usat}, interviews with experts in section \ref{expertinterview} and the focus group in section \ref{focusgroup} . 
The metrics varied in each method, SUS with experts and users returned a score which is also a number that represents how successful the user interface is, and the case study returned a time of task completion in seconds.
\begin{itemize}
    \item 7. What new knowledge is added to the knowledge base and in what form(e.g peer reviewed literature, meta-artifacts , new theory, new method.) ?
\end{itemize}
The artifact fit with friends is implemented as a high-fidelity prototype and this master thesis is added as new knowledge to the knowledge base. 
\begin{itemize}
    \item 8. Has the research questions been satisfactorily addressed?
\end{itemize}
The research questions are addressed at the end of this chapter in section \ref{researchQ}. The questions are answered and the information is presented with details and references from this thesis. 


\section{Design Principles}
The design principles were used to improve and focus on the usability of the application. Design principles focuses on design over functionality to achieve an intuitive user interface. The principles are used to enhance the design and are crucial in having continuous development with an easy to use design.

\section{Data Gathering}

\subsection{Literature review}
A literature review was conducted in the first design iteration to gather data. The literature review laid the foundation for the research project and helped in understanding what theoretical and practical work that have previously been done. The literature review showed that technology and social functionality can help people improve their fitness, but that the current research is limited and more research can be done to get more specific information about what users prefer. 
\subsection{Survey}
The online survey was conducted in the first design iteration and was completed by potential users. The survey provided a lot of information in a short amount of time. The information that was gathered was useful to understand what functionality and data users prefer from their fitness applications and devices.
\subsection{Semi-structured interviews}
Semi-structured interviews were used in this project to gather qualitative data and were important to establishing the proof of concept. The experts were online personal trainers and both gave a lot of suggestions and improvements for the application and how information should be shared. The pre-defined questions were important to start the discussions and it allowed for follow up questions. A challenge with interviewing experts was time, as they are busy and had limited time, having a workshop with them, would have been a great addition to the project.

\subsection{Focus group}
The focus group was valuable as the information that was gathered came directly from the intended users,a group of friends who lack motivation to workout and want to increase their physical activity. The challenges from the focus group was that two of the members took control and made the most suggestions and led the discussions. The other members did not voice their opinions in some of the discussions and one of the members was quiet for most of the discussion. 


\subsection{Case study}
The case study was used to gather data and test for the intuitiveness of the application. The test members were observed while interaction with the prototype and performed a SUS analysis as well. The case study provided insight on how users would interact with the functionality and what changes could be done for future designs.

\section{Evaluation of Prototypes}

\subsection{Usability testing}
Usability testing gives an indication on how intuitive the user interface design is. Usability testing with an intended target group provided useful insight on how potential users would interact with the application. The two users were given three tasks each, which they completed without any help given. This implies that the application`s user interface is easy and clear to understand. 
\subsection{System usability scale}
Usability experts and target group users performed SUS analysis as a quick and easy method to determine and evaluate the usability of the application. SUS allowed for the application to be tested by many users, which gave valuable feedback and some of the issues were highlighted in the feedback portion of the analysis. Both groups gave positive results which reflects that the design is user focused.

\subsection{Nielsen`s heuristics}
Usability experts performed the Nielsen`s heuristics evaluation in the last iteration of the prototype development.  The experts discovered issues with the application that could have been better designed and developed.  The heuristic evaluation allowed for the application to be tested by experts in a short amount of time, however the evaluation should have been performed in an earlier iteration as the prototype development would have benefited more from it. The usability experts answered that the application was satisfactory  !!!!!!
\section{Prototype Development}
Having four design iterations allowed for the application to be evaluated by multiple experts and intended users. Using low-fidelity prototypes in the beginning was useful to visualize the design and test out potential functionality and receive quick feedback on different ideas. The mid-fidelity and high-fidelity prototypes allowed for testing the functionality that was created, and see how the users would interact with the application. There were many smaller iterations between each main design iteration, which were helpful to have continuous improvements and discussions with co-students about the design and functionality of the application. 

\section{Limitations}
The research project had a few limitations throughout the process. Time was the biggest limitation as more design iterations would be beneficial for the users. The Corona virus halted the user testing too, more users could have been tested. However because of the quarantine, it was not possible to continue testing the design of the application and this limited the amount of iterations for the project.  The users could be more involved in the design process and having them evaluate the application in an extended amount of time to truly test the social network aspects of the application and have them test it in a more natural setting. Another limitation was learning react during the development phase, as it took some time to learn react and understand how to implement and create components.Fixing usability problems and more functionality could have been implemented,  however they are transferred to future work due to time limitations. 

\section{Research Questions} \label{researchQ}
The research questions will now be discussed:

\textbf{RQ 1}


\textbf{RQ 2}
