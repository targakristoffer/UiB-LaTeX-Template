\chapter{Sedentary lifestyle and exercise}
This chapter highlights what exercise and strength training is, how it can be done and the benefits of exercising. The chapter also looks at the different problems that can arise from living a sedentary lifestyle. What motivation is in terms of exercise, how users can be inspired and motivated.

\section{Benefits of exercise}
There are many benefits of working out and increasing your physical activity for health. As a sedentary lifestyle may lead to lifestyle diseases like cardiovascular problems and diabetes, staying active and in shape has benefits to everything from decreasing the chances of lifestyle diseases to improving mental health, sleep and cognition. Exercise has positive influences on cognitive abilities and sleep quality improves with exercise \cite{Lurati}. Workers, who are physically fit, usually have a low resting heart rate.

\section{Problems of sedentary lifestyle}
More problems
Besides the physical problems of being sedentary, there are other problems that are related to being inactive,people who has experiences with weight stigma were related to lower self-esteem, increased depression and increased body dissatisfaction \cite{Friedman}.
A person who lives a sedentary lifestyle might be afraid of going to the gym \cite{Vartanian2008}. Embarrassment caused by actual or anticipated negative evaluations from others might motivate some people to actively avoid public exercise situations, such as fitness centers and swimming pools. This is a common way of thinking where people are afraid of doing exercises the wrong way and are afraid that others will make fun of them. A way of eliminating this issue is to increase the knowledge among potential gym goers and people who live a sedentary lifestyle, encouraging them through technology to increase their own knowledge about exercise and physical fitness.

\section{Daily steps and cardiovascular health}
As regular physical activity is suggested as an important method to prevent cardiovascular diseases. There have been suggestions made that only walking is enough to stay physical fit and that people who walk 10000 steps a day has less chances of having cardiovascular risk factors. The correlation have been verified that people who walked more than 10000 steps a day had less body mass index, body fat percentage and triglycerides. Whereas the people who did not walk 10000 steps a day had greater chances of being overweight and dyslipemedia \cite{Rodrigues}. Thus concluding that there is an association between daily steps and cardiovascular risk factors. However it is not enough to remain physically fit as regular exercise is needed as well \cite{WHO}.

\section{Exercising and strength training}
A way of staying physically active is to exercise and do strength training, strength training is exercise that develops the strength and endurance of large muscle groups. It is also called “resistance training” or “weight training\cite{Resistance}. Strength training increases the fitness level, muscular strength, endurance, changes the body composition and more power. Even for frail elders it is understood that exercise and resistance training is positive. Supervised and controlled resistance training represents an effective intervention in frailty treatment \cite{Lopesz}.

By progressively overloading the weights that are lifted, or increasing repetitions or sets of exercises, people have a quantitative way of logging the progression. By setting short term or long-term goals like wanting to lift 100 kg in bench press makes it possible to both see progression, as you get stronger and stay motivated as you have a long-term goal.
if people want to stay active, motivated and encouraged it is good to get a partner. Exercising with a friend or relative can make it more fun. An exercise partner can offer support and encouragement \cite{FriendWorkout}. Exercising with friends and family will hold you accountable, Also, you will be less likely to skip a day of exercise if someone else is counting on you. And that when you work out, you should vary your routine. You are less likely to get bored or injured if you have some variety in your exercise routine. Keep track of your exercise to stay motivated. Use an app on your phone or a wearable activity tracker. You can even just mark a calendar with a checkmark each day you exercise \cite{FriendWorkout}.
\section{Motivation and exercise}
Motivation is a key feature of exercise and can account for individual differences in behaviours, inspiring people to engage in exercise \cite{Patel2015}.
Motivation for exercise can be divided into two categories, intrinsic motivation and extrinsic motivation. “Intrinsically motivated actions are experiences of competence, interest and enjoyment” \cite{Richard}, these are desires to engage new challenges and expand the skills.  “Extrinsic motivated behavior are those that are performed in order to obtain rewards or outcomes that are separate from the behavior itself” (ibid). These are body related motives, improving the appearance by losing weight or gaining more muscle, or fitness related motives like lifting more weight in the deadlift.  

A study done by Kilpatrick et al\cite{Kilpatrick} researched the motivation for physical activity for college students and reported that “results indicate that participants were more likely to report intrinsic motives, such as enjoyment and challenge, for engaging in sport,  whereas motivations for exercise were  more extrinsic and focused on appearance and weight and stress management. Since exercise has increasingly become a programmed activity \cite{Richard} they made the suggestion that making exercise or physical activity more intrinsically motivating(fun, personal challenging) might be a viable route to enhance persistence.” Another motivation for exercising is the increase in positive affect, the positive feelings about yourself, being able to relax and having increased energy, exercising gives the exerciser an enhanced sense of self in a pressure-free environment\cite{Frederick}.
