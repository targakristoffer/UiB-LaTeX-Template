\chapter{Requirements}
This chapter includes the ethical considerations for the project and the appropriate approvals that was secured from the Norwegian Centre for Research Data. The chapter presents the target group, the users that participated in testing the project, the experts... The requirements that were gathered from analysing stuff
\section{Ethical Considerations}
All participants were informed of their rights to be removed from the research project at any time and that their privacy is respected and secured. The participants were not asked any sensitive questions about their private life. 

The research was approved by the Norwegian Centre for Research Data (Norsk Senter for Forskningsdata - NSD). The participants signed an inform consent before being interviewed, testing or evaluating. The approval from NSD can be found in appendix .. Inform consent and interview guides can be found in appendix ..
\section{Target Group}
The target group is young adults between the ages of 18-30, who lives a sedentary lifestyle or exercise rarely and want to change. This was to focus on potential users who want to make adjustments and improve to maintain an active and healthy lifestyle. For the research it is important that the target group were interested in technology and were willing to use new applications. 

\begin{table}[H]
\centering
\begin{tabular}{ |l|l| }
  \hline
  \multicolumn{2}{|c|}{Target group} \\
  \hline
Gender & male/female\\ 
\hline
Age & 18-50\\
\hline
IT Criteria & smartphone, active on social media\\ 
  \hline
\end{tabular}
\label{tab:3}
\caption{Target group requirements}
\end{table}


\section{Research Participants}
\subsection{Users}
The users were recruited through personal connections. The users in the focus group which consisted of  one female and three males. Two of the users work in banking and investment, one masters student and one consultant. The focus group members averages one to two workout a week. 
The users in the case study and usability testing were two females, both are business and economics students and usually workout once or twice a week.
\subsection{Physical exercise experts}
The physical exercise experts consisted of a researcher in physiotherapy, a personal trainer and a former personal trainer who works as a developer. They were recruited through personal connections and took part in semi-structured interviews.
\subsection{Usability experts}
Six usability experts from the University of Bergen were recruited to evaluate the application with  Nielsen`s heuristics or SUS. The experts were one female and five males. The usability experts are all information science master students. The usability experts have a varied workout background, from being professional athletes that workout five to six times a week to living a sedentary lifestyle with no workouts. 
\section{Requirements}\label{requirements}
To establish requirements it is important to know who the users are, what features to implement and how to implement them. The requirements for a system are descriptions of services that a system should provide and constraints on its operation \cite{Sommerville:2010:SE:1841764}. The requirements are the needs of the features and are often divided in to functional requirements and non-functional requirements.
\subsection{Functional Requirements}
Functional requirements are the statements of services the system should provide, how the system should react to particular inputs, and how the system should behave in particular situations \cite{Sommerville:2010:SE:1841764}.
For the functional requirements it is important to understand what the user needs from the application. A survey on social media was conducted for the purpose of analysing what features users need.  

\textbf{The application needs to}
\begin{itemize}
\item store information the users want to share
\item let users select goals
\item display common goals for the groups
\item compete and cooperate towards goals
\item be able to see their own progress and group members progress
\item log workouts

\end{itemize}
    
\subsection{Non-Functional Requirements}
The non-functional requirements are the constraints on the services or functions offered by the system. The non-functional requirements include timing constraints, constraints on the development process and constraints imposed by standards \cite{Sommerville:2010:SE:1841764} . These requirements apply to the system as a whole rather than individual features. 

\textbf{The interface needs to}
\begin{itemize}
\item be user-friendly (fast responding, lean design)
\item be responsive
\item be aesthetically pleasing with modern design
\item be designed within the delivery of this paper(1st june 2020)
\end{itemize}
    