\chapter{Evaluation}
This chapter presents the evaluation results of the different design iterations from the SUS, usability testing and Nielsen`s heuristics.

\section{Participants}
Different groups evaluated each design iteration and the project switched between usability experts and intended users. The first group that participated in a focus group consisted of intended users who were a group of friends with a similar fitness level and were all motivated to improving their fitness levels. The other groups were usability experts who have all gotten a IT related degree and have participated in courses with interaction design and human-computer interaction. The two users that participated in the last design iteration are two friends that are business students and are frequent users of fitness applications (fitbit).

\section{System Usability Scale}
The usability experts evaluated the application first with a system usability scale (SUS) method. Both the experts and intended users evaluated the prototype on a laptop screen with an iPhone X layout. 

\section{SUS with Experts}\label{suse}
The usability experts did the SUS evaluation during the third design iteration with a mid-fidelity prototype with some functionality implemented. The usability experts returned an average score of 92. The experts answered the survey individually and were in the same room.
\begin{table}[H]
\centering
\begin{tabular}{ |l|l| }
  \hline
  \multicolumn{2}{|c|}{Expert SUS test} \\
  \hline
Group 1 & 90,8\\ 
\hline
Group 2 & 93,3\\
\hline
\end{tabular}
\caption{Results from experts testing}
\end{table}
\section{SUS with Users}\label{susu}
The users did a SUS evaluation during the fourth design iteration with a mid to high-fidelity prototype with most of the functionality working correctly. The users returned an average score of 91. The users were tested individually and on different days.
\begin{table}[H]
\centering
\begin{tabular}{ |l|l| }
  \hline
  \multicolumn{2}{|c|}{Intended user test} \\
  \hline
User 1 & 92\\ 
\hline
User 2 & 90\\
\hline
\end{tabular}
\caption{Results from user testing}
\end{table}

\section{Usability Testing}\label{usat}
Two users were presented with four different tasks and were timed to find out how effective and efficiently they could use the prototype. The users were not given any instructions on how to solve a task, however they had prior knowledge about the prototype as they were introduced to the research project and were presented a quick walk through of the prototype. The users completed the four tasks with no help or guidance. One noticeable difference between the users was that one user tried to navigate the prototype by clicking on the image as the user thought it would be similar to a home button to the main page.

Task 1 was to find a prior gym workout where the user trained bench press, the users started from the starting point of the prototype. User 2 was a bit slower on this test, however she explained that it took some time to read each of the mocked workout logs.
\begin{table}[H]
\centering
\begin{tabular}{ |l|l| }
  \hline
  \multicolumn{2}{|c|}{Task 1} \\
  \hline
User 1 & 4,2\\ 
\hline
User 2 & 7,6\\
\hline
\end{tabular}
\caption{Task 1 results}
\end{table}

Task 2 was to add a workout that the users could recall they did recently. The users started from the previous workout section. They were given some time before the test started to think of what they would like to log. Both users were similar in time to log their workouts.
\begin{table}[H]
\centering
\begin{tabular}{ |l|l| }
  \hline
  \multicolumn{2}{|c|}{Task 2} \\
  \hline
User 1 & 25,6\\ 
\hline
User 2 & 22,7\\
\hline
\end{tabular}
\caption{Task 2 results}
\end{table}

Task 3 was to find a friends(named Ola) workout where he did a 5 km run. The users started from their own workout log. User 2 struggled initially as she went below the leaderboard first before she scrolled up again and clicked on the expandable list of her friends workouts. 

\begin{table}[H]
\centering
\begin{tabular}{ |l|l| }
  \hline
  \multicolumn{2}{|c|}{Task 3} \\
  \hline
User 1 & 6,5\\ 
\hline
User 2 & 10,3\\
\hline
\end{tabular}
\caption{Task 3 results}
\end{table}

Task 4 was to check the leaderboard and see who is currently in third place, the users were told to navigate back to their own workout log before the timing started. Both users knew from beforehand where the leaderboard was and were very quick to report who was in third place.

\begin{table}[H]
\centering
\begin{tabular}{ |l|l| }
  \hline
  \multicolumn{2}{|c|}{Task 4} \\
  \hline
User 1 & 3,9\\ 
\hline
User 2 & 3,2\\
\hline
\end{tabular}
\caption{Task 4 results}
\end{table}

\section{Heuristics with experts}\label{heur}
Nielsen`s heuristics was used as the last step of the fourth design iteration.Three information science masters students evaluated and tested the high-fidelity prototype. The evaluators were presented the prototype on a laptop with iPhone X layout and evaluated the prototype separately. Below is the condensed feedback for all of the ten heuristics. The feedback is divided into four different rates.
\begin{itemize}
    \item Cosmetic usability problem: Need not to be fixed unless extra time is available on the project.
    \item Minor usability problem: fixing this should be given a low priority.
    \item Major usability problem: Important to fix, should have a high priority.
    \item Usability catastrophe: Imperative to fix, product can not be released.
\end{itemize}


\textbf{1. Visibility of System Status:}
Minor usability problem, there is no home screen, there could be a home screen on used on the heading banner. \\

\textbf{2. Match between the System and the Real World}
Cosmetic usability problem, in the navigation bar there is little match between icons and real life besides the group icon, the other two should be changed to something that specifies logging a workout and my workouts \\

\textbf{3. User control and Freedom}
Minor usability problem, there is no emergency exit, however since there are error prevention in place of logging workouts, this is not a big issue \\

\textbf{4. Consistency and Standard}
Cosmetic usability problem, the previous and next icons on the leaderboard should be horizontal and not vertical \\

\textbf{5. Error prevention}
Minor usability problem, can not delete training logs,even though the application asks for confirmation while logging workouts, it should be possible to delete the workouts later on \\

\textbf{6. Recognition rather than Recall}
Experts thought the applications information was presented so the users would not have to memorize it. \\

\textbf{7. Flexibility and Efficiency of Use}
The application is suitable for both experienced and inexperienced users. \\

\textbf{8. Aesthetics and Minimalistic Design}
Cosmetic usability problem, the header is hard to read, change the design \\

\textbf{9. Help Users Recognise, Diagnose, and Recover from Errors} 
Major usability problem, when an error occurred there were no error message, must have error messages to explain what the users can do  \\

\textbf{10. Help and Documentation}
Minor usability problem, no help or documentation, but the application is pretty easy to use and understand so should not have a high priority to add this \\

\textbf{Other usability problem(s)} 
Major usability problem, can not change goals 


