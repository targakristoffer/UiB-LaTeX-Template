\chapter{Conclusion and Future Work}
\section{Conclusion}
This project has contributed with a high-fidelity prototype called fit with friends. Throughout this project the design science research methodology was used to verify relevance, rigor, quality, as well as the design of the artifact. From the user and expert evaluation of the artifact, the results can be considered as meaningful and novel contribution to the existing knowledge base.

The application is committed to increasing the physical activity for sedentary people. Information was gathered by doing a literature review and data from a survey. The data was analyzed and helped establish the core requirements for the application. The application had four design iterations where each phase had user testing or evaluation and the feedback was collected to improve the design and functionality for every iteration.  The application started as a low-fidelity prototype drawn on paper and proto.io and ended up as a high-fidelity prototype created in the framework ReactJS. The high-fidelity prototype was interactive and tested by experts and users.

Interviews with experts in personal training helped form the requirements for the prototype development. The first low-fidelity prototype was tested by usability experts with SUS to reassure the design was easy to use and learn. The feedback from the experts was used to create a mid-fidelity prototype in ReactJS. The prototype was evaluated in a focus group consisting of a group of friends. The focus group was useful to understand how friends would interact with each other through the application, the feedback was used to add more functionality and some minor changes to the design. Next phase had a high-fidelity prototype with most of the functionality working. The prototype was evaluated with SUS and tested by experts, where the experts gave a good usability score. Afterwards the experts answered a group survey to ask if there were any functionality missing, how they would like to interact with the leaderboard and whether or not they would use a social fitness application for the long term. The experts also performed a heuristic evaluation in the last iteration to review the intuitiveness of the user interface, the feedback here is the basis of the future work.
Two users performed a usability test and SUS. The users gave a high usability score and performed the tasks without any mistakes. 

Fit with friends and the functionality is a recommendation for current fitness applications. Users rapidly abandon their fitness devices and applications and are overwhelmed with data, leaderboards with thousands of users and complicated design. This project shows that users are interested in simple functionality and social features where they can interact with friends. Competing with friends and co-operating by sharing information will help users stay active and accountable.  
\section{Future Work}
\subsection{Maintaining fit with friends}
Maintaining an application is important for the application to survive and stay relevant. As more technological devices are created for the fitness industry to gather data and information it is important to find a valuable way to represent the data. The functionality from fit with friends could be incorporated with  popular fitness application to make them social which would let the users interact more with each other. This would be the best case to let developers implement the features and use their own design. The application should be available in Google play store or iTunes to make the functionality visible to potential users. If the application was to be deployed, the feedback from the expert users with Nielsen`s heuristics for the design and functionality should be considered. The users should be able to delete training logs, change their goals and error messages should be available for the users to explain what happened and what the users can do.

\subsection{New features}

groups,
chat, 
diet,
caloric estimates,
cookie cutter programs

